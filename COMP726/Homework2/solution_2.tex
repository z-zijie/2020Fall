\documentclass{article}
% --- Modify margins --- %
\usepackage{geometry}
\geometry{a4paper,scale=0.8}
% --- Involved packages --- %
\usepackage{amssymb}
\usepackage{amsthm}
\usepackage{amsmath}

% --- Title information --- %
\title{CS 726 - Fall 2020\\
        {\Large \textbf{Homework \#2}}\\
        {\normalsize \textbf{Due : 10/05/2020, 5pm}}
    }
\author{Zijie Zhang}
\date{\today}


% --- main --- %

\begin{document}
    \maketitle

% --- Question 1 --- %
\section*{Question 1}
    \begin{proof}
        $\Rightarrow$,
        Let
        $$\varphi(\alpha)=\frac{1}{\alpha}\left(f\left((1-\alpha)x+\alpha y\right)-f(x)\right)$$
        $f$ is $m$-strongly convex means,
        $$f\left((1-\alpha)x+\alpha y\right) \leqslant (1-\alpha)f(x)+\alpha f(y)-\frac{m}{2} \alpha (1-\alpha)||y-x||^2$$
        $$f\left((1-\alpha)x+\alpha y\right) - f(x) \leqslant \alpha\left(f(y)-f(x)\right)-\frac{m}{2}\alpha(1-\alpha)||y-x||^2$$
        $$f(y)-f(x) \geqslant \varphi(\alpha)+\frac{m}{2}(1-\alpha)||y-x||^2$$
        Let $\alpha \to 0$,  we have
        $$f(y) \geqslant f(x) + \varphi^{'}(0)+\frac{m}{2}||y-x||^2=f(x) + \langle \nabla f(x), y-x\rangle + \frac{m}{2}||y-x||^2$$
        $$f(x+\alpha(y-x)) \geqslant f(x) + \langle \nabla f(x), \alpha(y-x)\rangle + \frac{m}{2}\alpha^2||y-x||^2$$
        Consider, Taylor Theorem:
        $$f(x+\alpha(y-x)) = f(x)+\langle \nabla f(x), \alpha(y-x)\rangle + \frac{\alpha^2}{2}(y-x)^T \nabla^2f(x+\gamma\alpha (y-x))(y-x)$$
        Combine the above two formulas, it gives
        $$(y-x)^T\nabla^2 f(x) (y-x) \geqslant m ||y-x||^2$$
        Thus, we have
        $$\nabla^2 f(x) \succeq mI$$

        $\Leftarrow$,
        By Taylor Theorem,
        $$f(y)=f(x)+\langle \nabla f(x), y-x \rangle + \frac{1}{2}\nabla^2 f(x+\gamma(y-x))||y-x||^2$$
        $\nabla^2 f(x) \succeq mI$ means the smallest eigenvalue of $\nabla^2 f(x)$ is greater than $m$, therefore
        $$\frac{1}{2}\nabla^2 f(x+\gamma(y-x))||y-x||^2 \geqslant \frac{1}{2}m||y-x||^2$$
        That is $$f(y) \geqslant f(x) + \langle \nabla f(x), y-x\rangle + \frac{m}{2}||y-x||^2$$
        Consider $$(1-\alpha)f(x)+\alpha f(y) -f((1-\alpha)x+\alpha y)$$
        We will have $$(1-\alpha)f(x)+\alpha f(y) -f((1-\alpha)x+\alpha y) \geqslant \frac{m}{2}||y-x||^2(\alpha -\alpha^2)=\frac{m}{2}\alpha(1-\alpha)||y-x||^2$$
    \end{proof}

% --- Question 2 --- %
\section*{Question 2}
    \begin{proof}
    Let $x_{k+1}=x_{k}+\nabla f(x_k)$, we have
    $$f(x_{k+1})-f(x_k)\geqslant \langle \nabla f(x_k),\nabla f(x_k) \rangle + \frac{m}{2}||\nabla f(x_k)||^2$$
    Add them together,
    $$f(x_{k+1})-f(x_0) \geqslant \left(1+\frac{m}{2}\right)\sum_{i=0}^{k} ||\nabla f(x_{i})||^2 \geqslant \left(1+\frac{m}{2}\right) \left\lVert \sum_{i=0}^{k} \nabla f(x_{i})\right\rVert^2 = \left(1+\frac{m}{2}\right)\left\lVert x_{k+1}-x_0 \right \rVert^2$$
    The gradient of $f$ can go to $\infty$, when $\left\lVert x_{k+1}-x_0 \right \rVert$ is large enough.\\
    So, $f$ cannot be Lipschitz continuous on the entire $\mathbb{R}^d$.
    But it is possible on the unit Euclidean ball.
    \end{proof}


% --- Question 3 --- %
\section*{Question 3}
    \begin{proof}
        
    \end{proof}


% --- Question 4 --- %
\section*{Question 4}
    \begin{proof}
        
    \end{proof}


% --- Question 5 --- %
\section*{Question 5}
    \begin{proof}
        
    \end{proof}

\end{document}