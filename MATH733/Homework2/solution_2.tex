\documentclass{article}
% --- Modify margins --- %
\usepackage{geometry}
\geometry{a4paper,scale=0.8}
% --- Involved packages --- %
\usepackage{amssymb}
\usepackage{amsthm}
\usepackage{amsmath}

% --- Title information --- %
\title{Math 733 - Fall 2020\\
        {\Large \textbf{Homework 2}}\\
        {\normalsize \textbf{Due: 09/27, 10pm}}
    }
\author{Zijie Zhang}
\date{\today}


% --- main --- %

\begin{document}
    \maketitle
    \begin{enumerate}
        % === Question 1 ===
        \item \begin{proof}
            Assume $X$ and $Y$ are random variables from $\left(\Omega, \mathcal{F}\right)$
            to $(S,\mathcal{S})$, then
            $$\{\omega:f\left(X(\omega)\right)\in B\}
            =\{\omega:X(\omega) \in f^{-1}(B)\}
            \in \mathcal{F}$$
            If and only if $f^{-1}(B) \in \sigma(X)$,
            $Y = f(X)$ is measurable w.r.t $\sigma(X)$.
        \end{proof}

        % === Question 2 ===
        \item \begin{proof}
            \begin{align*}
                E[X^p] &= \int_{0}^{1} y^p \cdot \mathbb{P}(X=y) dy\\
                    &= \int_{0}^{1-\varepsilon} y^p \cdot \mathbb{P}(X=y) dy
                        + \int_{1-\varepsilon}^{1} y^p \cdot \mathbb{P}(X=y) dy
                        \ \forall \varepsilon \in [0,1]\\
            \end{align*}
            When $p\to\infty$,
            $$\int_{0}^{1-\varepsilon} y^p \cdot \mathbb{P}(X=y) dy \to 0$$
            \begin{align*}
                \int_{1-\varepsilon}^{1} y^p \cdot \mathbb{P}(X=y) dy
                &\leqslant \int_{1-\varepsilon}^{1} y^p dy\\
                &=\frac{1}{p+1}\cdot \left(1-(1-\varepsilon)^p\right)\\
                &\leqslant \frac{1}{1+p} \to 0 \ (as \ p\to \infty)
            \end{align*}
            So, $E\left[X^p\right]=0$ as $p\to \infty$.
        \end{proof}

        % === Question 3 ===
        \item \begin{proof}
            \indent
            \begin{itemize}
                \item[(a)]
                    Consider \textit{Derangement formula}
                    $$\mathbb{P}(X_n=0)=\frac{D(n)}{n!}$$
                    where $$D(n)=n!\cdot \sum_{k=2}^{n}\frac{(-1)^k}{k!}$$
                    Then $$\mathbb{P}(X_n=0)=\sum_{k=2}^{n}\frac{(-1)^k}{k!}$$
                    When $n\to \infty$,
                    $$\mathbb{P}(X_n=0)\to\sum_{k=2}^{\infty}\frac{(-1)^k}{k!}$$
                    This is one of the expression of $\frac{1}{e}$.
                    $$\lim_{n\to\infty} \mathbb{P}(X_n=0)=\frac{1}{e}$$
                \item[(b)]
                    Noticed that
                    $$\mathbb{P}(X_n=1) = \frac{{n \choose 1}D(n-1)}{n!}=\frac{D(n-1)}{(n-1)!}=\mathbb{P}(X_{n-1}=0)$$
                    $$\mathbb{P}(X_n=2) = \frac{{n \choose 2}D(n-2)}{n!}=\frac{1}{2}\cdot\frac{D(n-2)}{(n-2)!}=\frac{1}{2}\mathbb{P}(X_{n}=0)$$
                    $$\mathbb{P}(X_n=k) = \frac{{n \choose k}D(n-k)}{n!}=\frac{1}{k}\mathbb{P}(X_n=k-1)$$
                    So, we have
                    $$\mathbb{P}(X_n=k)=\frac{1}{k!}\cdot \mathbb{P}(X_n=0)$$
                    \begin{align*}
                        E[X_n]&=\sum_{k=0}^n k\cdot \mathbb{P}(X_n=k)\\
                        &=\sum_{k=0}^n k\cdot \frac{1}{k!}\mathbb{P}(X_n=0)\\
                        &=\mathbb{P}(X_n=0)\cdot\sum_{k=1}^n \frac{1}{(k-1)!}\\
                        &=\sum_{k=2}^n \frac{(-1)^k}{k!}\cdot \sum_{k=1}^n\frac{1}{(k-1)!}
                    \end{align*}
                    By the way, when $n$ is large enough, $E[X_n] \to 1$.
            \end{itemize}
        \end{proof}

        % === Question 4 ===
        \item \begin{proof}
            Consider the Integral form of \textit{Cauchy–Schwarz inequality}.\\
            Let $f = y \cdot \sqrt{\mathbb{P}(Y=y)}$, $g=\sqrt{\mathbb{P}(Y=y)}$.\\
            By the non-negativity of $Y$
            $$\left(\int\left(y\cdot \sqrt{\mathbb{P}(Y=y)}\right)^2 dy\right)
            \cdot \left(
            \int\left(\sqrt{\mathbb{P}(Y=y)}\right)^2 dy\right)
            \geqslant \left(\int y\cdot \mathbb{P}(Y=y) dy \right)^2$$
            That is
            $$\mathbb{P}(Y>0) \geqslant \frac{\left(E[Y]\right)^2}{E[Y^2]}$$
        \end{proof}

        % === Question 5 ===
        \item \begin{proof}
            When $X_1=X_2=\cdots=X_n=S$, $\prod_{j=1}^n g_j =1$\\
            Consider
            $$\mathbb{P}\left(X_1=x_1, X_2 = S, \cdots, X_n=S\right)
            =g_1(x_1) \cdot C \ \text{    C is a constant}$$
            So we have
            $$\mathbb{P}\left(X_2 = x_2, \cdots, X_n=x_n|X_1=x_1\right)
            =\frac{\prod_{j=2}^n g_j(x_j)}{C}$$
            Thus, $X_1$ is independent.
            So does $X_2, \cdots, X_n$.
            On the other hand, if they are independent, we have
            $$\mathbb{P}\left(X_1=x_1, \cdots, X_n=x_n\right)
            =\prod_{j=1}^n \mathbb{P}\left(X_j=x_j\right)$$
            Here, let $g_j(x) = \mathbb{P}\left(X_j=x\right)$ for all $1\leqslant j \leqslant n$, they are non-negative functions from $S$ to $\mathbb{R}$.
        \end{proof}

        % === Question 6 ===
        \item \begin{proof}
            Consider the expression of $\omega$ in binary. The probability of $X_n=1$
            means the probability that the $n$th decimal place is 1.
            By the definition and arbitrariness of $\omega$, we know
            $$\mathbb{P}\left(X_n=0\right)=\mathbb{P}\left(X_n=1\right)=\frac{1}{2}$$
            That means, $$X_n \sim \text{Bernoulli(1/2)}, \ \forall n\in \mathbb{N}$$
            And their independence is obvious.
        \end{proof}
    
    \end{enumerate}
\end{document}