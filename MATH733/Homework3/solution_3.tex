\documentclass{article}
% --- Modify margins --- %
\usepackage{geometry}
\geometry{a4paper,scale=0.8}
% --- Involved packages --- %
\usepackage{amssymb}
\usepackage{amsthm}
\usepackage{amsmath}
\usepackage{mathrsfs}
\usepackage{bbm}

% --- Title information --- %
\title{Math 733 - Fall 2020\\
        {\Large \textbf{Homework 3}}\\
        {\normalsize \textbf{Due: 10/11, 10pm}}
    }
\author{Zijie Zhang}
\date{\today}


% --- main --- %

\begin{document}
    \maketitle
    \begin{enumerate}
        % === Question 1 ===
        \item 
        \begin{enumerate}
        \item \begin{proof} 
            $$X\sim B(n,p) \Rightarrow P(X=k)=\binom{n}{k}p^k(1-p)^{n-k}$$
            $$Y\sim B(m,p) \Rightarrow P(X=k)=\binom{m}{k}p^k(1-p)^{m-k}$$
            Then \begin{align*}
                P(X+Y=k)&=\sum_{i=0}^k P(X=i, Y=k-i)\\
                &=\sum_{i=0}^k P(X=i)\cdot P(Y=k-i)\\
                &=\sum_{i=0}^k \binom{n}{i}p^i(1-p)^{n-i}
                    \cdot \binom{m}{k-i}p^{k-i}(1-p)^{m-k+i}\\
                &=p^k(1-p)^{m+n-k}\sum_{i=0}^k \binom{n}{i}\binom{m}{k-i}\\
                &=\binom{n+m}{k}p^k(1-p)^{m+n-k}
            \end{align*}
            Thus, $$X+Y \sim B(n+m,p)$$
        \end{proof}
        \item \begin{proof}
            $$X\sim \text{Poisson}(\lambda) \Rightarrow P(X=k)=\frac{\lambda^k}{k!}e^{-\lambda}$$
            $$Y\sim \text{Poisson}(\mu) \Rightarrow P(Y=k)=\frac{\mu^k}{k!}e^{-\mu}$$
            Then \begin{align*}
                P(X+Y=k)&=\sum_{i=0}^k P(X=i, Y=k-i)\\
                &=\sum_{i=0}^k P(X=i)\cdot P(Y=k-i)\\
                &=\sum_{i=0}^k \frac{\lambda^i}{i!}e^{-\lambda}
                    \cdot \frac{\mu^{k-i}}{(k-i)!}e^{-\mu}\\
                &=e^{-(\lambda+\mu)}\sum_{i=0}^k \frac{\lambda^i}{i!}\frac{\mu^{k-i}}{(k-i)!}\\
                &=\frac{(\lambda+\mu)^k}{k!}e^{-(\lambda+\mu)}
            \end{align*}
            Thus, $$X+Y \sim \text{Poisson}(\lambda+\mu)$$
        \end{proof}
        \end{enumerate}


        % === Question 2 ===
        \item \begin{proof}
            
        \end{proof}


        % === Question 3 ===
        \item \begin{proof}
            
        \end{proof}

        

        % === Question 4 ===
        \item \begin{proof}
            
        \end{proof}


        % === Question 5 ===
        \item \begin{proof}

        \end{proof}


        % === Question 6 ===
        \item \begin{proof}
            
        \end{proof}
    
    \end{enumerate}
\end{document}