\documentclass{article}
% --- Modify margins --- %
\usepackage{geometry}
\geometry{a4paper,scale=0.8}
% --- Involved packages --- %
\usepackage{amssymb}
\usepackage{amsthm}
\usepackage{amsmath}
\usepackage{enumerate}

% --- Title information --- %
\title{CS 726 - Fall 2020\\
        {\Large \textbf{Homework \#2}}\\
        {\normalsize \textbf{Due : 10/05/2020, 5pm}}
    }
\author{Zijie Zhang}
\date{\today}


% --- main --- %

\begin{document}
    \maketitle

% --- Question 1 --- %
\section*{Question 1}
    \begin{proof}
        $\Rightarrow$,
        Let
        $$\varphi(\alpha)=\frac{1}{\alpha}\left(f\left((1-\alpha)x+\alpha y\right)-f(x)\right)$$
        $f$ is $m$-strongly convex means,
        $$f\left((1-\alpha)x+\alpha y\right) \leqslant (1-\alpha)f(x)+\alpha f(y)-\frac{m}{2} \alpha (1-\alpha)||y-x||^2$$
        $$f\left((1-\alpha)x+\alpha y\right) - f(x) \leqslant \alpha\left(f(y)-f(x)\right)-\frac{m}{2}\alpha(1-\alpha)||y-x||^2$$
        $$f(y)-f(x) \geqslant \varphi(\alpha)+\frac{m}{2}(1-\alpha)||y-x||^2$$
        Let $\alpha \to 0$,  we have
        $$f(y) \geqslant f(x) + \varphi^{'}(0)+\frac{m}{2}||y-x||^2=f(x) + \langle \nabla f(x), y-x\rangle + \frac{m}{2}||y-x||^2$$
        $$f(x+\alpha(y-x)) \geqslant f(x) + \langle \nabla f(x), \alpha(y-x)\rangle + \frac{m}{2}\alpha^2||y-x||^2$$
        Consider, Taylor Theorem:
        $$f(x+\alpha(y-x)) = f(x)+\langle \nabla f(x), \alpha(y-x)\rangle + \frac{\alpha^2}{2}(y-x)^T \nabla^2f(x+\gamma\alpha (y-x))(y-x)$$
        Combine the above two formulas, it gives
        $$(y-x)^T\nabla^2 f(x) (y-x) \geqslant m ||y-x||^2$$
        Thus, we have
        $$\nabla^2 f(x) \succeq mI$$

        $\Leftarrow$,
        By Taylor Theorem,
        $$f(y)=f(x)+\langle \nabla f(x), y-x \rangle + \frac{1}{2}\nabla^2 f(x+\gamma(y-x))||y-x||^2$$
        $\nabla^2 f(x) \succeq mI$ means the smallest eigenvalue of $\nabla^2 f(x)$ is greater than $m$, therefore
        $$\frac{1}{2}\nabla^2 f(x+\gamma(y-x))||y-x||^2 \geqslant \frac{1}{2}m||y-x||^2$$
        That is $$f(y) \geqslant f(x) + \langle \nabla f(x), y-x\rangle + \frac{m}{2}||y-x||^2$$
        Consider $$(1-\alpha)f(x)+\alpha f(y) -f((1-\alpha)x+\alpha y)$$
        We will have $$(1-\alpha)f(x)+\alpha f(y) -f((1-\alpha)x+\alpha y) \geqslant \frac{m}{2}||y-x||^2(\alpha -\alpha^2)=\frac{m}{2}\alpha(1-\alpha)||y-x||^2$$
    \end{proof}

% --- Question 2 --- %
\section*{Question 2}
    \begin{proof}
    Let $x_{k+1}=x_{k}+\nabla f(x_k)$, we have
    $$f(x_{k+1})-f(x_k)\geqslant \langle \nabla f(x_k),\nabla f(x_k) \rangle + \frac{m}{2}||\nabla f(x_k)||^2$$
    Add them together,
    $$f(x_{k+1})-f(x_0) \geqslant \left(1+\frac{m}{2}\right)\sum_{i=0}^{k} ||\nabla f(x_{i})||^2 \geqslant \left(1+\frac{m}{2}\right) \left\lVert \sum_{i=0}^{k} \nabla f(x_{i})\right\rVert^2 = \left(1+\frac{m}{2}\right)\left\lVert x_{k+1}-x_0 \right \rVert^2$$
    The gradient of $f$ can go to $\infty$, when $\left\lVert x_{k+1}-x_0 \right \rVert$ is large enough.\\
    So, $f$ cannot be Lipschitz continuous on the entire $\mathbb{R}^d$.
    But it is possible on the unit Euclidean ball.
    \end{proof}


% --- Question 3 --- %
\section*{Question 3}
    \begin{proof}
        By Lemma2.2
        \begin{align*}
            f(x_{k+1})
            &\leqslant f(x_k) 
                + \left\langle \nabla f(x_k), x_{k+1}-x_k \right\rangle 
                + \frac{L}{2} \left\lVert x_{k+1}-x_k \right \rVert_2^2\\
            &=f(x_k)
                -\alpha_k \left\langle \nabla f(x_k), \nabla_{i_k}f(x_k)e_{i_k} \right\rangle
                +\frac{L}{2} \alpha_k^2 \left\lVert \nabla_{i_k}f(x_k)e_{i_k} \right \rVert_2^2\\
            &=f(x_k)
                +\left(\frac{L}{2}\alpha_k -1\right)\alpha_k \left\lVert \nabla_{i_k}f(x_k)e_{i_k} \right \rVert_2^2
        \end{align*}
        Choose $\alpha_k = \frac{1+\sqrt{1-L\beta d^2}}{L}$, then
        $$\mathbb{E}[f(x_{k+1})-f(x_k)] = -\frac{\beta d^2}{2}\mathbb{E}[\left\lVert \nabla_{i_k}f(x_k)e_{i_k} \right \rVert_2^2]=-\frac{\beta}{2}\left\lVert \nabla f(x_k) \right \rVert_2^2$$
    \end{proof}


% --- Question 4 --- %
\section*{Question 4}
    \begin{proof}
        \indent
        \begin{enumerate}[(i)]
            \item $$\nabla \psi(y)=y-x_0$$
            \begin{align*}
                D_{\psi}(x,y)
                &=\frac{1}{2}\left\lVert x-x_0 \right \rVert_2^2
                    -\frac{1}{2}\left\lVert y-x_0 \right \rVert_2^2
                    -\left\langle \nabla \psi(y), x-y \right\rangle\\
                &=\frac{1}{2}\left\lVert x \right \rVert_2^2
                    +\frac{1}{2}\left\lVert y \right \rVert_2^2
                    -\left\langle x,y \right\rangle\\
                &=\frac{1}{2}\left\lVert x-y \right \rVert_2^2
            \end{align*}
            \item $$\nabla \phi(y)=\psi(y)+x_0=y$$
            \begin{align*}
                D_{\phi}(x,y)
                    &=\phi(x)-\phi(y)-\langle y,x-y\rangle\\
                    &=\psi(x)-\psi(y)+\langle x_0-y,x-y \rangle\\
                    &=\frac{1}{2}\left\lVert x \right \rVert_2^2
                    +\frac{1}{2}\left\lVert y \right \rVert_2^2
                    -\left\langle x,y \right\rangle\\
                    &=D_{\psi}(x,y)
            \end{align*}
            \item Left: $D_{\psi}(x,y) = \frac{1}{2}\left\lVert x \right \rVert_2^2
            +\frac{1}{2}\left\lVert y \right \rVert_2^2
            -\left\langle x,y \right\rangle $\\
            Right: \begin{align*}
                &D_{\psi}(z,y)
                    +\left\langle \nabla\psi(z)-\nabla\psi(y), x-z \right\rangle
                    +D_{\psi}(x,z)\\
                &=\frac{1}{2}\left\lVert z \right \rVert_2^2
                    +\frac{1}{2}\left\lVert y \right \rVert_2^2
                    -\langle z,y \rangle
                    +\langle z-y,x-z \rangle
                    +\frac{1}{2}\left\lVert x \right \rVert_2^2
                    +\frac{1}{2}\left\lVert z \right \rVert_2^2
                    -\langle x,z \rangle\\
                &=\frac{1}{2}\left\lVert x \right \rVert_2^2
                +\frac{1}{2}\left\lVert y \right \rVert_2^2
                -\left\langle x,y \right\rangle
            \end{align*}
            \begin{center}
                Left=Right
            \end{center}
            \item Obviously, $\nabla m_k(v_k)=0$, thus
            \begin{align*}
                D_{m_k}(x,v_k)&=m_k(x)-m_k(v_k)\\
                &=\sum_{i=0}^k a_i D_{\psi_i}(x,v_k)\\
                &=\sum_{i=0}^k a_i \left(
                    \frac{1}{2}\left\lVert x \right \rVert_2^2
                    +\frac{1}{2}\left\lVert v_k \right \rVert_2^2
                    -\langle x,v_k \rangle
                    \right)\\
                &=\sum_{i=0}^k a_i \frac{1}{2}\left\lVert x-v_k \right \rVert_2^2\\
                &=\frac{A_k}{2}\left\lVert x-v_k \right \rVert_2^2
            \end{align*}
            So, we have proved
            $$m_{k+1}(x)=m_k(v_k)+a_{k+1}\psi_{k+1}(x)+\frac{A_k}{2}\left\lVert x-v_k \right \rVert_2^2$$
        \end{enumerate}
    \end{proof}


% --- Question 5 --- %
\section*{Question 5}
    \begin{proof}
        \begin{itemize}
            \item Let $\nabla h_z(x)=0$, that is
            $$ z+\nabla\left(\frac{1}{2}\left\lVert x \right \rVert_p^2\right)=0$$
            For every $i$,
            $$z_i = -x_i^{p-1}\left(\sum_{i=1}^d x_i^p\right)^{\frac{2-p}{p}}$$
            We need find $\left(\sum_{i=1}^d x_i^p\right)^{\frac{2-p}{p}}$ in $z$.
            Calculate $\left\lVert z \right \rVert_q^2$, we have
            $$\left(\sum_{i=1}^d z_i^q\right)^{\frac{1}{q}}=-\left(\sum_{i=1}^d x_i^{p}\right)^{\frac{1}{q}}\left(\sum_{i=1}^d x_i^p\right)^{\frac{2-p}{p}}=-\left(\sum_{i=1}^d x_i^p\right)^{\frac{1}{p}}$$
            So, for every $i$, we have
            $$x_i=-z_i^{q-1}\left(\sum_{i=1}^d z_i^q\right)^{\frac{2-q}{q}}$$
            Which is, $$x=-\nabla\left(\frac{1}{2}\left\lVert z \right \rVert_q^2\right)$$
            Substituting x into $h_z$ gives
            \begin{align*}
                h_z\left(-\nabla\left(\frac{1}{2}\left\lVert z \right \rVert_q^2\right)\right)
                &=\left\langle z,-\nabla\left(\frac{1}{2}\left\lVert z \right \rVert_q^2\right) \right\rangle+\frac{1}{2}\left\lVert -\nabla\left(\frac{1}{2}\left\lVert z \right \rVert_q^2\right) \right \rVert_p^2\\
                &=-\left\lVert z \right \rVert_q^2+\frac{1}{2}\left(\sum_{i}^d\left(-z_i^{q-1}\left\lVert z \right \rVert_q^{2-q}\right)^p\right)^{\frac{2}{p}}\\
                &=-\left\lVert z \right \rVert_q^2+\frac{1}{2}\left\lVert z \right \rVert_q^2\\
                &=-\frac{1}{2}\left\lVert z \right \rVert_q^2
            \end{align*}
            \item Let $$z=\frac{1}{L}\nabla f(x_k),\ x=u-x_k$$
                We know $$f(x_{k+1})\leqslant f(x_k)-\frac{1}{2}\left\lVert \frac{1}{L}\nabla f(x_k) \right \rVert_q^2=f(x_k)-\frac{1}{2L}\left\lVert \nabla f(x_k) \right \rVert_q^2$$
            \item \begin{align*}
                f(x_{k+1}) &\leqslant f(x_k)-\frac{1}{2L}
                    \left\lVert \nabla f(x_k) \right \rVert_q^2\\
                    &\leqslant f(x_{k-1})-\frac{1}{2L}
                    \left\lVert \nabla f(x_{k-1}) \right \rVert_q^2
                    -\frac{1}{2L}
                    \left\lVert \nabla f(x_{k}) \right \rVert_q^2\\
                    &\leqslant \cdots\\
                    &\leqslant f(x_{0})-\frac{1}{2L}
                    \sum_{i=0}^k \left\lVert \nabla f(x_{i}) \right \rVert_q^2\\
            \end{align*}
            Assume $f(x)\geqslant f_{*} \geqslant -\infty$($f$ is bounded below), then
            \begin{align*}
                &\frac{1}{2L}(k+1) \left(
                    \min_{0\leqslant i \leqslant k}\left\lVert \nabla f(x_{i}) \right \rVert_q\right)^2\\
                &\leqslant \frac{1}{2L}
                \sum_{i=0}^k \left\lVert \nabla f(x_{i}) \right \rVert_q^2\\
                &\leqslant f(x_0)-f(x_{k+1})\\
                &\leqslant f(x_0)-f_{*}
            \end{align*}
            Therefore, we have
            $$\min_{0\leqslant i \leqslant k}\left\lVert \nabla f(x_{i}) \right \rVert_q \leqslant \sqrt{\left(f(x_0)-f_{*}\right)\frac{2L}{k+1}}$$
        \end{itemize}
    \end{proof}

\end{document}