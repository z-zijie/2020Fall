\documentclass{article}
% --- Modify margins --- %
\usepackage{geometry}
\geometry{a4paper,scale=0.8}
% --- Involved packages --- %
\usepackage{amssymb}
\usepackage{amsthm}
\usepackage{amsmath}
\usepackage{mathrsfs}
\usepackage{bbm}
\usepackage{graphicx}
\usepackage{listings}
\usepackage{enumitem}
% \usepackage{enumerate}
\numberwithin{equation}{section}
\renewcommand\thesection{\Alph{section}}
\counterwithin{figure}{section}
\renewcommand{\thefigure}{\arabic{section}.\arabic{figure}}

% --- Title information --- %
\title{Math 714 - Fall 2020\\
        {\Large \textbf{Homework 2}}
    }
\date{}


% --- main --- %

\begin{document}
    \maketitle
    \section{}
    \begin{enumerate}[label=(\alph*)]
        \item If $v\in \text{span}\{w_1,\cdots, w_n\}$, then there exists $\{\alpha_j\}_{j=1}^n such that $$$v = \sum_{j=1}^{n} \alpha_j w_j$$
        $$\frac{\langle v, w_j\rangle}{\Vert w_j \Vert^2} = \frac{\alpha_j \langle w_j, w_j \rangle}{\Vert w_j \Vert^2} = \alpha_j$$
        $$v = \sum_{j=1}^n \alpha_j w_j = \sum_{j=1}^n \frac{\langle v, w_j\rangle}{\Vert w_j \Vert^2}w_j$$
        \item ii. For $n=1$, $p_1 = r_1 - \frac{\langle r_1,p_0 \rangle}{\Vert p_0\Vert^2}p_0$.
        \begin{align*}
            \langle p_1, p_0 \rangle
            & = \langle r_1 - \frac{\langle r_1,p_0 \rangle}{\Vert p_0\Vert^2}p_0, p_0 \rangle\\
            & = \langle r_1, p_0 \rangle - \frac{\langle r_1,p_0 \rangle}{\Vert p_0\Vert^2}\Vert p_0\Vert^2\\
            & = 0
        \end{align*}
        Suppose $n=k$ is true. When $n = k+1$, for $0 \leqslant j \leqslant k$,
        \begin{align*}
            \langle p_{k+1}, p_j \rangle
            & = \langle r_{k+1}, p_j \rangle - \frac{\langle r_{k+1}, p_j \rangle}{\Vert p_j \Vert^2} \langle p_j, p_j \rangle\\
            & = 0
        \end{align*}
        By induction, it is true for all $0 \leqslant j < n \leqslant n^*-1$.
        \item \begin{enumerate}[label=\roman*]
            \item \begin{align*}
                \langle Av, w \rangle
                & = \sum_{n=1}^N \langle v, \phi_n \rangle \langle A\phi_n, w \rangle\\
                & = \sum_{n=1}^N \langle v, \phi_n \rangle \langle \lambda_n\phi_n, w \rangle\\
                & = \sum_{n=1}^N \lambda_n\langle v, \phi_n \rangle \langle \phi_n, w \rangle
            \end{align*}
            \item By the definition of positive definite matrix.
            \item $v = \sum_{n=1}^N\alpha_n \phi_n$,  then $\langle Av,v \rangle = \sum_{n=1}^N \lambda_n \alpha_n^2$, $\Vert v \Vert^2 = \sum_{n=1}^N \alpha_n^2$.\\
            By $\lambda_1 \leqslant \cdots \leqslant \lambda_N$,  we know
            $$\sum_{n=1}^N \lambda_1 \alpha_n^2 \leqslant \sum_{n=1}^N \lambda_n \alpha_n^2 \leqslant \sum_{n=1}^N \lambda_N \alpha_n^2$$
            $$\lambda_1 \Vert v \Vert^2 \leqslant \langle Av, v\rangle \leqslant \lambda_N \Vert v \Vert^2$$
            \item $$\Vert Av\Vert^2 = \langle Av, Av \rangle = \sum_{n=1}^N \alpha_n^2 \lambda_n^2 \leqslant \sum_{n=1}^N \alpha_n^2 \lambda_N^2 = \lambda_N^2 \Vert v \Vert^2 $$
            $$\Vert Av \Vert \leqslant \lambda_N \Vert v \Vert$$
        \end{enumerate}
        \item \begin{align*}
            p_{n+1}
            & = r_{n+1} + \beta_n p_n\\
            & = r_n -\alpha_n \omega_n + \beta_n p_n\\
            & = r_n - \alpha_n A p_n + \beta_n p_n\\
            & = p_n - \beta_{n-1}p_{n-1} - \alpha_n A p_n + \beta_n p_n
        \end{align*}
        Thus, $$p_{n+1} = (1+\beta_n)p_n - \alpha_n A p_n - \beta_{n-1}p_{n-1}$$
        \item By Cayley-Hamilton theorem, $p(\lambda) = \det(\lambda I -A)$
        $$p(\lambda) = A^N + \alpha_{N-1}A^{N-1} + \cdot + \alpha_1 A + (-1)^N \det |A|I_N = 0$$
        Thus $A^N$ is a linear combination of $I, A, A^2, \cdot, A^{N-1}$.
        \item \begin{enumerate}[label=\roman*]
            \item \begin{align*}
                e_n
                & = u_n - u\\
                & = u_{n-1} + \alpha(f-Au_{n-1})-u\\
                & = (I-\alpha A) (u_{n-1}-u)\\
                & = (I-\alpha A)e_{n-1}
            \end{align*}
            \item \begin{align*}
                \Vert e_{n+1}\Vert
                & = \Vert(I-\alpha A)e_{n}\Vert\\
                & \leqslant \Vert (I-\alpha A)\Vert\cdot \Vert e_n \Vert
            \end{align*}
            Noticed that $$\rho(A) = \max_{1\leqslant i \leqslant N}|\lambda_i|$$
            then, $$\Vert e_{n+1}\Vert \leqslant \rho \Vert e_{n}\Vert$$
            where $\rho = \max_{1\leqslant j \leqslant N}|1-\alpha \lambda_j|$
            \item \begin{align*}
                \rho 
                & = \max_{1\leqslant j \leqslant N}|1-\alpha \lambda_j|\\
                & = \max(|1-\alpha \lambda_1|, |1-\alpha \lambda_N|)
            \end{align*}
            $$1-\alpha\lambda_1 = -1+\alpha\lambda_N$$
            $$\alpha = \frac{2}{\lambda_1+\lambda_N}$$
            Thus,  we have $\rho = 1 - \frac{2\lambda_1}{\lambda_1+\lambda_N} = \frac{\kappa-1}{\kappa+1} < 1$, where $\kappa = \frac{\lambda_N}{\lambda_1}$.
            \item \begin{align*}
                \rho
                & = \max_{1\leqslant j \leqslant N}|1-\alpha \lambda_j|\\
                & = \max\left(\left|1-\frac{2\lambda_1}{c+C}\right|, \left|1-\frac{2\lambda_N}{c+C}\right|\right)
            \end{align*}
            Noticed that \begin{align*}
                \left|1-\frac{2\lambda_1}{c+C}\right|
                & \leqslant \frac{C-c}{C+c}\\
                1-\frac{2C}{C+c} \leqslant 1-&\frac{2\lambda_1}{C+c} \leqslant 1-\frac{2c}{C+c}\\
                \Leftarrow c \leqslant &\lambda_1 \leqslant C
            \end{align*}
            Then, $$\rho \leqslant \frac{C-c}{C+c} = \frac{\kappa'-1}{\kappa' +1} <1$$
        \end{enumerate}
        \item \begin{enumerate}[label=\roman*]
            \item $$r_1 = r_0 - \alpha_0 A p_0 = r_0 - \alpha_0 A r_0$$
            \item \begin{align*}
                r_{n+1}
                & = r_n - \alpha_n A p_n\\
                & = r_n - \alpha_n A (r_n + \beta_{n-1}p_{n-1})\\
                & = r_n - \alpha_n A r_n - \alpha_n\beta_{n-1}\frac{r_{n-1}-r_n}{\alpha_{n-1}}
            \end{align*}
            \item \begin{align*}
                r_0q_0 - \delta_0q_1
                & = \frac{1}{\alpha_0}q_0 - \frac{\sqrt{\beta_0}}{\alpha_0}q_1\\
                & = \frac{1}{\alpha_0}q_0 - \frac{r_1}{\alpha_0\Vert r_0 \Vert}\\
                & = \frac{\alpha_0Ar_0}{\alpha_0 \Vert r_0 \Vert}\\
                & = Aq_0
            \end{align*}
            $$\frac{r_{n+1}}{\Vert r_n\Vert} = q_n - \alpha_n A r_n +\frac{\alpha_n \beta_{n-1}}{\alpha_{n-1}}\left(q_n-\frac{r_{n-1}}{\Vert r_n\Vert}\right)$$
            $$q_{n+1}\sqrt{\beta_n} = q_n -\alpha_n A q_n + \frac{\alpha_n \beta_{n-1}}{\alpha_{n-1}}\left(q_n - \frac{q_{n-1}}{\sqrt{\beta_{n-1}}}\right)$$
            $$Aq_n = -\delta_{n-1}q_{n-1}+\gamma_n q_n - \delta_n q_{n+1}$$
            \item 
            \item $$Q_n^TAQ_n = Q_n^T(Q_nT_n-\delta_{n-1}q_ne_n^T) = T_n$$
        \end{enumerate}
    \end{enumerate}

\end{document}