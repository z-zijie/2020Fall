\documentclass{article}
% --- Modify margins --- %
\usepackage{geometry}
\geometry{a4paper,scale=0.8}
% --- Involved packages --- %
\usepackage{amssymb}
\usepackage{amsthm}
\usepackage{amsmath}
\usepackage{bm}
\usepackage{graphicx}
\usepackage{physics}
\usepackage{enumitem}
\usepackage{listings}
\usepackage{xcolor}
\usepackage[utf8]{inputenc}
\usepackage[english]{babel}

\renewcommand{\familydefault}{\sfdefault}

\definecolor{codegreen}{rgb}{0,0.6,0}
\definecolor{codegray}{rgb}{0.5,0.5,0.5}
\definecolor{codepurple}{rgb}{0.58,0,0.82}
\definecolor{backcolour}{rgb}{0.95,0.95,0.92}

\lstdefinestyle{mystyle}{
    % backgroundcolor=\color{backcolour},   
    commentstyle=\color{codegreen},
    keywordstyle=\color{magenta},
    numberstyle=\tiny\color{codegray},
    stringstyle=\color{codepurple},
    basicstyle=\sffamily\footnotesize,
    breakatwhitespace=false,         
    breaklines=true,                 
    captionpos=b,                    
    keepspaces=true,                 
    % numbers=left,                    
    numbersep=5pt,                  
    showspaces=false,                
    showstringspaces=false,
    showtabs=false,                  
    tabsize=2
}

\lstset{style=mystyle}

% --- NewCommand --- %
% \newcommand{\norm}[1]{\left\lVert#1\right\rVert}

\usepackage{hyperref}
\hypersetup{
    colorlinks=true,
    linkcolor=blue,
    filecolor=magenta,      
    urlcolor=cyan,
}

\urlstyle{same}




% --- Title information --- %
\title{CS 760: Machine Learning - Fall 2020\\
        {\Large \textbf{Homework 4: Decision Trees}}\\
        {\normalsize \textbf{Due : 10/29/2020}}
    }
\author{Zijie Zhang}
\date{\today}


% --- main --- %

\begin{document}
    \maketitle
    
\begin{center}
    \Large
    All the code involved is in this \href{https://github.com/z-zijie/2020Fall/tree/master/COMP760/Homework4}{GitHub repository}.\\
    \url{https://github.com/z-zijie/2020Fall/tree/master/COMP760/Homework4}\\
    All the code is in this \href{https://github.com/z-zijie/2020Fall/blob/master/COMP760/Homework4/code.py}{python file}.\\
    \large
    \url{https://github.com/z-zijie/2020Fall/blob/master/COMP760/Homework4/code.py}
\end{center}

% --- Problem 1 ---%
\section*{Problem 1}
    \begin{itemize}
        \item \textbf{Pclass:}\\
        1 $\to$ [1, 0, 0];\\
        2 $\to$ [0, 1, 0];\\
        3 $\to$ [0, 0, 1].
        \item \textbf{Sex:}\\
        0 $\to$ [1, 0];\\
        1 $\to$ [0, 1].
        \item \textbf{Age:} Binning in [0, 20.25, 40.5, 60.75, 81] then convert to \textbf{One-Hot}.
        \item \textbf{Siblings/Spouses Aboard:} Convert to One-Hot like \textbf{Sex}.
        \item \textbf{Parents/Children Aboard:} Convert to One-Hot like \textbf{Sex}.
        \item \textbf{Fare:} Binning in [0,10,20,30,40,50,70,115,160,205,250,295,340,385,430,475,520] then convert to \textbf{One-Hot}.
    \end{itemize}
    \textbf{Explanation:}
    \begin{enumerate}
        \item Pclass, Sex, Siblings/Spouses Aboard and Parents/Children Aboard are \textbf{CATEGORICAL FEATURES}, so we can directly encode them One-Hot.
        \item Age and Fare are \textbf{NUMERICAL FEATURES}, we should binning them first. It is worth noticing that the distribution of \textit{Fare} is \textbf{skewed distribution}, we can't binning it uniformly.
    \end{enumerate}

% --- Problem 2 ---%
\section*{Problem 2}
    \begin{center}
        Please check the relevant \href{https://github.com/z-zijie/2020Fall/blob/master/COMP760/Homework4/code.py}{code} in the file.
    \end{center}
\pagebreak

% --- Problem 3 ---%
\section*{Problem 3}
    \begin{center}
        Please check the relevant \href{https://github.com/z-zijie/2020Fall/blob/master/COMP760/Homework4/code.py}{code} in the file.
    \end{center}
    \textbf{Stopping criteria} are
    \begin{enumerate}
        \item \textbf{If the entropy of the response y in the current subset of data is close to zero.}
        \item \textbf{If the current subset of data contains too few samples. (less than 5\% of the total data)}
        \item \textbf{If the depth of the Tree greater than the number of features.}
    \end{enumerate}\

% --- Problem 4 ---%
\section*{Problem 4}
    \begin{center}
        Please check the relevant \href{https://github.com/z-zijie/2020Fall/blob/master/COMP760/Homework4/code.py}{code} and output in the file.
    \end{center}
    \lstinputlisting{P4.txt}

\end{document}