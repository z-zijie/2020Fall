\documentclass{article}
% --- Modify margins --- %
\usepackage{geometry}
\geometry{a4paper,scale=0.8}
% --- Involved packages --- %
\usepackage{amssymb}
\usepackage{amsthm}
\usepackage{amsmath}

% --- Title information --- %
\title{Math 733 - Fall 2020\\
        {\Large \textbf{Homework 1}}\\
        {\normalsize \textbf{Due: 09/13, 10pm}}
    }
\author{Zijie Zhang}
\date{\today}


% --- main --- %

\begin{document}
    \maketitle
    \begin{enumerate}
        \item \begin{proof}
            We noticed that $A \circ B = (A \cup B) \setminus (A \cap B)$. So it
            gives
            $$B \circ C \cup A \circ C = (A \cup B \cup C) 
                \setminus (A \cap B \cap C)$$
            this means
            $$B \circ C \cup A \circ C \supset A \circ B$$
            thus
            $$\mathbf{P}(B \circ C \cup A \circ C) 
                \geqslant \mathbf{P}(A \circ B)$$
            we know
            $$\mathbf{P}(B \circ C) + \mathbf{P}(A \circ C)
                \geqslant \mathbf{P}(B \circ C \cup A \circ C)$$
            so, we proved
            $$\mathbf{P}(A \circ B) \leqslant
                \mathbf{P}(B \circ C) + \mathbf{P}(A \circ C)$$
        \end{proof}

        \item \begin{proof}
            $\mathcal{F}$ is a $\sigma$-algebra satisfy
            \begin{itemize}
                \item[(i)] if $A \in \mathcal{F}$, then $A^c \in \mathcal{F}$
                \item[(ii)] if $A_i \in \mathcal{F}$ is countable sequence of sets, then $\cup_i A_i \in \mathcal{F}$. 
            \end{itemize}
            On the other hand, the measure $\mathbf{P}$ is a nonnegative countably additive set function.
            \begin{itemize}
                \item[(i)] $\mathbf{P}(A)=0$ if $A$ is countable, which means $A^c$ is not countable, $\mathbf{P}(A^c)=1$.\\
                            $\mathbf{P}(A) \geqslant \mathbf{\emptyset} = 0$ for all $A\in \mathcal{F}$, $\emptyset$ is countable.
                \item[(ii)] if $A_i\in \mathcal{F}$ is countable sequence of disjoint sets, then $\mathbf{P}(A_i)=0$ and $\cup_i A_i$ is countable.\\
                            So, $\mathbf{P}(\cup_i A_i) = \sum_i \mathbf{P}(A_i)$
            \end{itemize}
            $\emptyset$ is countable, so $\mathbf{P}(\mathbb{R}) = 1$. \\
            $(\Omega,\mathcal{F},\mathbf{P})$ is a probability space.
        \end{proof}

        \item \begin{proof}
            Let \begin{align*}
                B_1 &= A_1 - A_2 \cup A_3 \cup \cdots \cup A_n \\
                B_2 &= A_2 - A_1 \cup A_3 \cup \cdots \cup A_n \\
                    &\cdots \\
                B_n &= A_n - A_1 \cup A_2 \cup \cdots \cup A_{n-1}\\
                B_{n+1} &= A_1 \cap A_2 - A_3 \cup A_4 \cup \cdots \cup A_n\\
                    &\cdots \\
                B_{n+{n \choose 2}+1} &= A_1 \cap A_2 \cap A_3 - A_4 \cup A_5 \cup \cdots \cup A_n\\
                    &\cdots \\
                B_{2^n-1} &= A_1 \cap A_2 \cap \cdots A_n\\
                B_{2^n} &= \Omega - A_1 \cup A_2 \cup \cdots \cup A_n\\
            \end{align*}
            So, the $\sigma$-field generated by $\{A_1, \cdots, A_n\}$ is exactly the set of finite unions of the sets $B_i$.
        \end{proof}
    
        \item \begin{proof}
            By the definition of $\sigma$-field.\\
            \begin{itemize}
                \item[(i)] If $A \in \cap_{j \in J}\mathcal{F}_j$, we know $A \in \mathcal{F}_j$ for all $j \in J$.\\
                            Thus $A^c \in \mathcal{F}_j$ for all $j \in J$, then we have $A^c \in \cap_{j \in J}\mathcal{F}_j$.\
                \item[(ii)] If $A_i \in \cap_{j \in J}\mathcal{F}_j$ is a countable sequence sets, then $A_i \in \mathcal{F}_j$ for all $j \in J$.\\
                            We have $\cup_i A_i \in \mathcal{F}_j$ for all $j \in J$, thus $\cup_i A_i \in \cap_{j \in J}\mathcal{F}_j$.
            \end{itemize}
            So, $\cap_{j \in J}\mathcal{F}_j$ is also a sigma-field.
        \end{proof}

        \item \begin{proof}
            Let $\Omega = \mathbb{R}[0,1]$, $\mathcal{F}$ is the $\sigma$-algebra.\\
            $\mathbf{P}(A)$ is the measure of the set $A$.\\
            $\mathbf{P}$ is Lebesgue measure.\\
            Let $X=(\mathbb{R}-\mathbb{Q})\cap \Omega$.\\
            $X={q}$ is null set, so $\mathbf{P}(X=q)=0$.\\
            Let $A_i \in \mathbb{Q}[0,1]$ is a countable sequence of disjoint sets. By the definition, we know
            $\mathbf{P}(\mathbb{Q}[0,1])=\mathbf{P}(\cup_i A_i) = \sum_i \mathbf{P}(A_i) = 0$.
            So $\mathbf{P}(X \text{is irrational}) = 1 - \mathbf{P}(\mathbb{Q}[0,1]) = 1$.
        \end{proof}

        \item \begin{proof}
            Set $A_k$ as event "the coin flips $k,k+1, \cdots, 2k$ are all heads.".\\
            $$\mathbf{P}(A_k)=\frac{1}{2^{k+1}}$$
            Set $M$ as event "there will be no integer $n$ so that the coin flips $n,n+1,\cdots, 2n$ are all heads.".
            \begin{align*}
                \mathbf{P}(M^c) &\leqslant\sum_{k=1}^{\infty}\mathbf{P}(A_k)\\
                &=\frac{1}{2}
            \end{align*}
            $$\mathbf{P}(M) = 1-\mathbf{P}(M^c) \geqslant \frac{1}{2}$$
            So we get the positive lower bound.
        \end{proof}
    \end{enumerate}
\end{document}