\documentclass{article}
% --- Modify margins --- %
\usepackage{geometry}
\geometry{a4paper,scale=0.8}
% --- Make content have hyper link --- %
\usepackage{hyperref}
\hypersetup{
    colorlinks,
    citecolor=black,
    filecolor=black,
    linkcolor=black,
    urlcolor=black
}

% --- main --- %

% \title{Math/Stat 733 - Theory of Probability I.}
% \author{Zijie Zhang}

\begin{document}
    \begin{titlepage}
    \begin{center}
        \vspace*{1cm}
        
        \Huge
        Math/Stat 733

        \vspace{0.5cm}
        \Huge
        Theory of Probability I

        \vspace{0.5cm}
        \LARGE
        Notebook

        \vspace{1.5cm}
        \Large
        Zijie Zhang

        \vfill
                
        \vspace{0.8cm}
        
        \today
                
    \end{center}
\end{titlepage}
    \tableofcontents
    \newpage

%  ---2020/08/29/Saturday---Create Notebook.tex---  % 
    \begin{center}
        \footnotesize Updated on 2020/08/29
    \end{center}

    \section{Overview}
        \subsection{Basic Information}
        {
        \textbf{Meetings:} TR 1pm-2:15pm (online)\\
        \textbf{Instructor:} Benedek Valkó\\
        \textbf{Email:} valko@math.wisc.edu\\
        \textbf{Office hours:} Tu 4-5pm, F 3-4pm, or by appointment (online)
        }

        \subsection{Textbook}
        \begin{center}
            \Large
            \textit{Richard Durrett: Probability: Theory and Examples, 5th edition, 2019}
        \end{center}
        \indent \textbf{Extra Reading}
        \begin{itemize}
            \item \large Olav Kallenberg: Foundations of Modern Probability. 2nd edition, Springer, 2002
            \item \large William Feller. An introduction to probability theory and its applications. Vol. I. Third edition. John Wiley and Sons Inc., New York, 1968.
            \item \large David Williams. Probability with martingales. Cambridge Mathematical Textbooks. Cambridge University Press, Cambridge, 1991.
            \item \large Patrick Billingsley. Probability and measure. Wiley Series in Probability and Mathematical Statistics. John Wiley \& Sons Inc., New York, 1995.
        \end{itemize}

        \subsection{Course content}
        We cover selected portions of Chapters 1-4 of Durrett. This is a rough course outline:
        \begin{itemize}
            \item Weeks 1-2: Foundations, properties of probability spaces
            \item Weeks 3-5: Independence, 0-1 laws, strong law of large numbers
            \item Weeks 6-10: Characteristic functions, weak convergence and the central limit theorem
            \item Weeks 11-15: Conditional expectation, Martingales
        \end{itemize}

        The course continues in the spring semester as Math 734 covering topics 
        such as Markov chains, stationary processes, ergodic theory, and Brownian motion.

        \subsection{Evaluation}
        Course grades will be based on \textit{biweekly} \\
        \indent \textbf{home work assignments (25\%)},\\
        \indent \textbf{class participation (15\%)},\\
        \indent \textbf{a midterm exam (30\%)}\\
        \indent and the \textbf{final exam (30\%)}.\\
        (See the Canvas page for more information.)
        
        \begin{center}
            \footnotesize End of Update on 2020/08/29
        \end{center}
% --- End of Overview --- %


%  ---2020/09/03/Thursday--- %

\newpage
\begin{center}
    \footnotesize Updated on 2020/09/03
\end{center}

\section{Sep 3, Thursday}
    \subsection{Intro}
    Upload Homework 1 on Canvas before Sep 13.
    \begin{itemize}
        \item Mid exam and final exam, open book.
        \item Midterm is evening midterm.
        \item Textbook is \textit{Richard Durrett: Probability: Theory and Examples, 5th edition, 2019}.
        \item Notes will be uploaded.
    \end{itemize}

    \subsubsection{What is probability theory?}
    \textbf{Goal:} model uncertain events, quantify ...\\
    Simple examples with built in symmetry.\\
    \begin{itemize}
        \item Filp a fair coin.\\
                What's the probability of tails?\\
    \end{itemize}
    

    \subsection{Kolmogorov axioms, examples}
    

\begin{center}
    \footnotesize End of Update on 2020/09/03
\end{center}

%  ---End of 2020/09/03/Thursday--- % 

\end{document}