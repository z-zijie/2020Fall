\documentclass{article}
% --- Modify margins --- %
\usepackage{geometry}
\geometry{a4paper,scale=0.8}
% --- Involved packages --- %
\usepackage{amssymb}
\usepackage{amsthm}
\usepackage{amsmath}
\usepackage{bm}

% --- Title information --- %
\title{CS 760: Machine Learning - Fall 2020\\
        {\Large \textbf{Homework 1: Review}}\\
        {\normalsize \textbf{Due : 09/24/2020}}
    }
\author{Zijie Zhang}
\date{\today}


% --- main --- %

\begin{document}
    \maketitle
    
% --- Problem 1 ---%
\section*{Problem 1}
    \begin{proof}
        By the definition, $\mathbb{R}^D$ is a subspace if for every
        $a,\ b\in \mathbb{R}$ and every $\bm{u}, \bm{v} \in \mathbb{R}^D$,
        $a\bm{u}+b\bm{v} \in \mathbb{R}^D$.\\
        \indent It is trivial.
    \end{proof}

% --- Problem 2 ---%
\section*{Problem 2}
    \begin{proof}
        \indent
        \begin{itemize}
            \item[(a)]
                $\bm{x} = (-1,\cdots,-1) \in \mathbb{R}^D$. The element-wise square roots is
                $(i,\cdots,i) \not\in \mathbb{R}^D$.\\
                $\mathbb{R}^D$ is not closed under element-wise square roots.
            \item[(b)] 
                $\mathbb{C}^D$.
        \end{itemize}
    \end{proof}

% --- Problem 3 ---%
\section*{Problem 3}
    \begin{proof}
        For every element $\bm{x},\ \bm{y} \in \mathbb{U}$. They can be represented by a linear
        combination of $\bm{u}_1, \cdots, \bm{u}_R$.
        $$\bm{x} = \alpha_1 \bm{u}_1 + \cdots + \alpha_R \bm{u}_R$$
        $$\bm{y} = \beta_1 \bm{u}_1 + \cdots + \beta_R \bm{u}_R$$
        For every $a,\ b\in\mathbb{R}$,
        $$a\bm{x}+b\bm{y} = (a\alpha_1+b\beta_1)\bm{u}_1+\cdots+(a\alpha_R+b\beta_R)\bm{u}_R $$
        $a\bm{x}+b\bm{y} \in \mathbb{U}$, so $\mathbb{U}$ is a subspace.
    \end{proof}

% --- Problem 4 ---%
\section*{Problem 4}
    \begin{proof}
        
    \end{proof}

% --- Problem 5 ---%
\section*{Problem 5}
    \begin{proof}
        
    \end{proof}

% --- Problem 6 ---%
\section*{Problem 6}
    \begin{proof}
        
    \end{proof}

% --- Problem 7 ---%
\section*{Problem 7}
    \begin{proof}
        
    \end{proof}


\end{document}